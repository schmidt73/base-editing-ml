\documentclass[12pt]{article}

\usepackage{geometry}
\usepackage{parskip}
\usepackage{xparse}
\usepackage{xspace}
\usepackage{algorithm}
\usepackage{algorithmicx}
\usepackage{algpseudocode}
\usepackage{amsfonts,amsmath,amsthm,amssymb}
\usepackage{mathtools}

\geometry{margin=1in}

%%%
% KL-divergence
%
%   \kld{p}{q}
%%%

\DeclarePairedDelimiterX{\infdivx}[2]{(}{)}{%
  #1\;\delimsize|\delimsize|\;#2%
}
\newcommand{\kld}[2]{\ensuremath{D_{KL}\infdivx{#1}{#2}}\xspace}

\begin{document}

\section*{Background}
Sequence-to-sequence models aim to learn the mapping between a set of
source and target strings, $\{(x_i, y_i)\}$. Specifically, models such
as the Transformer learn a conditional probability distribution
$P_\theta$ that maximizes the conditional probability of the target
$y_i$, given the value $x_i$:

$$\max_{\theta} \prod_{i=1}^n{P_\theta[y_i | x_i]} = \max_{\theta}\sum_{i=1}^n{\log{P_\theta[y_i | x_i]}}$$

In the Transformer, this distribution is constructed in an
autoregressive fashion. That is, for a given training pair $(x, y)$
the output is matrix $Y : S \times |\Sigma|$ where $S \in \mathbb{N}$
is the output sequence length. And each entry $(k, \sigma)$ is defined
as,

$$Y_{k,\sigma} = P_\theta[\sigma | y_1\ldots y_{k-1}]$$

Note that because this matrix $Y$ is row-stochastic, we can use the
KL-divergence between the rows of this matrix and the target one-hot
encoding as our loss. For clarity, the target matrix
$T : S \times |\Sigma|$ is defined as

$$T_{k, \sigma} =
  \begin{cases}
    1 \text{ if } \sigma = y_k\\
    0 \text{ otherwise}
  \end{cases}
$$

In implementation, this target matrix may be smoothed to avoid
overconfident predictions. Many details of the implementation have
been glossed over for clarity.

\section*{Base Editing Problem}
Unfortunately, the base-editing prediction problem does not directly
map onto the aforementioned sequence-to-sequence paradigm. Instead of
a single target, we observe mappings between source strings and
probability distributions. Specifically, we have a set of samples of
the form $\{(x_i,P_{y_i})\}$ where $P_{y_i}$ denotes a discrete
distribution over strings. Thus we aim to learn a mapping $\phi$ from
$x$ that parameterizes the distribution such that
$P_{\phi(x_i)} \approx P_{y_i}$.

A natural loss function for this model is to maximize the similarity,
or minimize the difference, between the target and predicted
distributions. That is we find $\phi$ that minimizes,

$$\min_{\phi}\sum_{i=1}^n\kld{P_{y_i}}{P_{\phi(x_i)}}$$

There are several difficulties placing this objective in the context
of the sequence-to-sequence model.

First, each row $k$ of the matrix $Y$ conditions on previous
symbols. During training time, we feed these symbols in iteratively by
masking the attention matrix to not allow look-ahead. The issue with
this conditioning is that it does not allow us to compute the
probability of an abitrary sequence without re-evaluating the model on
the new sequence. For example, suppose we want to compute
$P_\theta[ATC | x]$ and $P_\theta[ACT | x]$ under our model. The first
and second rows of the matrix $Y$ can be shared between the two
predictions since they condition on the shared strings ``'' and
``A''. However, the final row has to be recomputed since in the former
case, $Y$ conditions on $AT$ whereas in the latter case, it conditions
on $AC$.

Similarly, KL-divergence between rows of the matrix compares the
conditional distributions of a single source-target pair. It does not
suffice. Suppose that we have an extremely simple case where the
support of the predictive distribution is simply two strings $y_1$ and
$y_2$ that have observed frequencies $f$ and $1-f$ on input
$x$. Ideally, the KL-divergence we want to compute is between the two
vectors $[f, 1-f]$ and $[\hat{f}, 1-\hat{f}]$. But to compute $f$ we
must walk through the conditional matrix $Y_1$ in a target-dependant
manner and take the product of the a single entry from each row (by
the Chain rule).

It would be straightforward to directly predict the desired
probability distribution. Because the sample space is finite, we could
do this by predicting a long vector where each entry corresponds to
the probability of a particular string. However, this vector would be
intractably large, and almost all entries would be 0.

\section*{Proposed Approach}

I have tried several approaches, including one similar to that of
BE-Hive, however they all failed. Herein, I will describe the most
recent (and semi-successful) approach that I have taken.

Pick a single input string $x$. Then, we observe a probability
distribution over strings $P_y$. We can visualize this distribution as
follows,
\begin{align*}
  y_1 &= y_{1,1}\ldots y_{1, k}, f_1\\
  &\;\;\vdots \notag \\
  y_n &= y_{n,n}\ldots y_{n, k}, f_n
\end{align*}
where $\sum_{i=1}^nf_i = 1$.

We take this set and convert it into $n$ conditional probability
matrices where the $j$-th row of the $i$-th matrix correspond to the
conditional probability of an 





\end{document}

